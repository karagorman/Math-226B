\documentclass[12pt]{article}
\usepackage{float}
\usepackage[dvipsnames]{xcolor}
\usepackage{dingbat, tikz}
\usepackage{amsfonts,amsmath, color, fullpage, graphicx, mathtools, empheq, amsthm, amssymb}
\usepackage{thmtools}
\usepackage{listings}

\renewcommand{\qedsymbol}{\begingroup \color{blue} \rightthumbsup \endgroup}
%\renewcommand{\qedsymbol}{\textcolor{\red}{\rightthumbsup}}

\def\C{\mathbb{C}}
\def\N{\mathbb{N}}
\def\Q{\mathbb{Q}}
\def\R{\mathbb{R}}
\def\Ts{\mathbb{T}}
\def\Z{\mathbb{Z}}
\def\T{\mathcal{T}}
\def\P{\mathcal{P}}
\title{Math 226B HW \#1}

\begin{document}

\begin{flushleft}
\textbf{Kara Gorman}\\
\textbf{Homework \#1}\\
\textbf{Math 226B}: \textbf{Winter 2018}
\end{flushleft}

\bigskip\bigskip
\noindent
\textbf{Problem 1:} Let $A \in \R^{n\times n}$ be a row-stochastic matrix, $\alpha \in \R$ a parameter, and
$$A_{\alpha} := \alpha A + (1 - \alpha)\frac{1}{n}ee^T, \text{ where } e:= \begin{bmatrix}
			1 \\
			1 \\
			\vdots \\
			1 \end{bmatrix} \in \R^n.$$
Prove that the matrix $A_{\alpha}$ is row-stochastic for all parameter values $0 \leq \alpha \leq 1$.\\

\begin{proof} $\text{ }$\\

\underline{Case 1:} $\alpha = 0$
$$A_0 = \frac{1}{n}ee^T,$$
where
\begin{align}
B :&= \frac{1}{n}ee^T \nonumber \\
&= \frac{1}{n} \begin{bmatrix}
		1 \\
		1 \\
		\vdots \\
		1 \end{bmatrix}
		\begin{bmatrix}
		1 & 1 & \hdots & 1
		\end{bmatrix} \nonumber \\
&= \frac{1}{n} \begin{bmatrix}
				1 & 1 & \hdots & 1 \\
				1 & 1 & \hdots & 1 \\
				\vdots & \vdots & \ddots & \vdots \\
				1 & 1 & \hdots & 1 
				\end{bmatrix} \nonumber \\
&= \begin{bmatrix}
	\frac{1}{n} & \frac{1}{n} & \hdots & \frac{1}{n} \\
	\frac{1}{n} & \frac{1}{n} & \hdots & \frac{1}{n} \\
	\vdots & \vdots & \ddots & \vdots \\
	\frac{1}{n} & \frac{1}{n} & \hdots & \frac{1}{n} 
	\end{bmatrix}  				
\end{align}

(1) is an $n\times n$ matrix where $\sum_{j=1}^n b_{ij} = 1$, for all $i=1, \hdots , n$, i.e., the sum of each row of matrix $B$ is equal to 1.  Thus, $A_0 = B$ is row-stochastic.\\

\underline{Case 2:} $\alpha = 1$\\
$$A_1 = A$$
Since we are given that $A$ is a row-stochastic matrix, then clearly $A_1$ is row-stochastic.\\

\underline{Case 3:} $0 < \alpha < 1$\\
$$A_{\alpha} = \alpha A + (1-\alpha)\frac{1}{n}ee^T,$$
where $A$ and $B = \frac{1}{n}ee^T$ are both row-stochastic matrices, as shown in cases 1 and 2.  So, we can write $A_{\alpha}$ as:

$$A_{\alpha} = \begin{bmatrix}
				\alpha a_{11}+(1-\alpha)b_{11} & \alpha a_{12}+(1-\alpha)b_{12} & \hdots & \alpha a_{1n}+(1-\alpha)b_{1n} \\
				\alpha a_{21}+(1-\alpha)b_{21} & \alpha a_{22}+(1-\alpha)b_{22} & \hdots & \alpha a_{2n}+(1-\alpha)b_{2n} \\
				\vdots & \vdots & \ddots & \vdots \\
				\vdots & \vdots & \ddots & \vdots \\
				\alpha a_{n1}+(1-\alpha)b_{n1} & \alpha a_{n2}+(1-\alpha)b_{n2} & \hdots & \alpha a_{nn}+(1-\alpha)b_{nn} 
				\end{bmatrix}. $$\\

If we look at the first row, we see that:
\begin{align}
\sum_{j=1}^n a^{\alpha}_{1j} &= [\alpha a_{11} + (1-\alpha)b_{11}] + 		[\alpha a_{12}+(1-\alpha)b_{12}] + \dots + [\alpha a_{1n}				+(1-	\alpha)b_{1n}] \nonumber \\
&= \alpha (a_{11} + a_{12} + \dots + a_{1n}) + (1-\alpha)(b_{11} + b_{12} + \dots + b_{1n}) \nonumber \\
&= \alpha(1) + (1-\alpha)(1) \nonumber \\
&= \alpha + 1 - \alpha \nonumber \\
&= 1 \nonumber
\end{align}

In general, for all $i = 1, \dots , n$,
\begin{align}
\sum_{j=1}^n a^{\alpha}_{ij} &= [\alpha a_{i1} + (1-\alpha)b_{i1}] + 		[\alpha a_{i2}+(1-\alpha)b_{i2}] + \hdots + [\alpha a_{in}			+(1-	\alpha)b_{in}] \nonumber \\
&= \alpha (a_{i1} + a_{i2} + \dots + a_{in}) + (1-\alpha)(b_{i1} + 		b_{i2} + \dots + b_{in}) \nonumber \\
&= \alpha(1) + (1-\alpha)(1) \nonumber \\
&= \alpha + 1 - \alpha \nonumber \\
&= 1 \nonumber
\end{align}

Thus, $A_{\alpha} = \alpha A + (1-\alpha)\frac{1}{n}ee^T$ is row-stochastic for all $\alpha$ such that $0 \leq \alpha \leq 1$.

\end{proof}

\newpage
\bigskip\bigskip
\noindent
\textbf{Problem 2:} A small company runs an internal network of 10 websites, which have no links to the outside world.  The links within the internal network are as follows:\\
\begin{itemize}
\item Website 1 has links to websites 5, 7;
\item Website 2 has links to websites 2, 6, 7, 9;
\item Website 4 has links to websites 4, 7;
\item Website 5 has links to websites 7, 9, 10;
\item Website 8 has links to websites 3, 4, 7, 8, 9;
\item Website 9 has links to websites 2, 4, 7, 8, 10;
\item Website 10 has links to websites 1, 3, 4, 7, 9, 10;
\end{itemize}
Formulate a linear algebra  problem, the solution of which is the PageRank vector x of this internal network, and compute x.\\

According to your computed PageRank, what is the ranking of the 10 websites from most to least important?\\

\begin{proof} $\text{ }$\\
First, we need to construct the matrix $Q$, the matrix associated with the directed graph of the connectivity of the 10 given websites.  Note that we do not include self-links in our $Q$ matrix.\\

$$Q = \begin{bmatrix}
		0&0&0&0&\frac{1}{2}&0&\frac{1}{2}&0&0&0 \\
		0&0&0&0&0&\frac{1}{3}&\frac{1}{3}&0&\frac{1}{3}&0 \\
		0&0&0&0&0&0&0&0&0&0 \\
		0&0&0&0&0&0&1&0&0&0 \\
		0&0&0&0&0&0&\frac{1}{3}&0&\frac{1}{3}&\frac{1}{3} \\
		0&0&0&0&0&0&0&0&0&0 \\
		0&0&0&0&0&0&0&0&0&0 \\
		0&0&\frac{1}{4}&\frac{1}{4}&0&0&\frac{1}{4}&0&\frac{1}{4}&0 \\
		0&\frac{1}{5}&0&\frac{1}{5}&0&0&\frac{1}{5}&\frac{1}{5}&0&\frac{1}{5} \\
		\frac{1}{5}&0&\frac{1}{5}&\frac{1}{5}&0&0&\frac{1}{5}&0&\frac{1}{5}&0 \\
		\end{bmatrix} $$\\

Now, we can construct the vector $v = [v_j]_{j=1,\dots,n} \in \R^n$, which is defined by
$$v_j = \begin{cases}
		1 \text{ if } d_j = 0 \\
		0 \text{ if } d_j > 0,
		\end{cases}
$$
where $d_j$ denotes the out degree of node $j$ of $Q$.  For this problem,\\
$$v = \begin{bmatrix}
		0\\
		0\\
		1\\
		0\\
		0\\
		1\\
		1\\
		0\\
		0\\
		0
		\end{bmatrix} $$\\

Now, we can use the following Matlab function to find the eigenvalue and eigenvectors of $A^T$, where\\
$$ A = Q + \frac{1}{n}ve^T.$$

\lstset{language=matlab,frame=single}
\begin{lstlisting}[caption=PageRank Function]
function [y] = PageRank(Q,v)
 
 n = length(v);
 e = ones(n,1);
 
 A = Q + (1/n)*v*transpose(e);
 
 AT = transpose(A);
 
 [V,D] = eig(AT);
 
 end
\end{lstlisting}
$\text{ }$ \\
This Matlab function produces the eigenvalues and eigenvectors of $A^T$.  To determine the page rankings, we identify the leading order eigenvalue, $\lambda = 1$, and then its corresponding eigenvector \\
$$u = \begin{bmatrix}
		-1.611685764370067e-01 \\
		-1.785251923609917e-01 \\
		-2.057998745272545e-01 \\
		-2.717395180598953e-01 \\
		-1.931698370468546e-01 \\
		-1.720939462820155e-01 \\
		-7.479616674742428e-01 \\
		-1.785251923609919e-01 \\
		-3.296982176632033e-01 \\
		-2.429151380432770e-01
	\end{bmatrix},$$\\

determines the page rankings of each website.  The website with the highest magnitude is the most important.  So, the wedsites in order from most to least important are:
$$7, 9, 4, 10, 3, 5, 8, 2, 6, 1.$$

\end{proof}

\bigskip\bigskip
\noindent
\textbf{Problem 3:} Let $A \in \R^{n\times n}$ be a row-stochastic matrix.\\

\begin{itemize}

\item[(a)] Show that all eigenvalues $\lambda$ of $A$ satisfy $|\lambda| \leq 1$.\\

\begin{proof} From the definition of row-stochasticity, we know that $A$ has the eigenvalue $|\lambda_1| = 1$ with the eigenvector 
$u_1 = \begin{bmatrix}
		1 \\
		1 \\
		\vdots \\
		1
		\end{bmatrix}$.\\

Now, assume there is an eigenvalue $|\lambda_2| > 1$ of $A$, with eigenvector $u_2$.  Then, $A^n u_2 = |\lambda_2|^n u_2$ grows exponentially as n increases.  However, this implies that for large $n$, there is an entry of matrix $A$ such that $a_{jk}^n > 1$.  But this is a contradiction since the product of two row-stochastic matrices is also a row-stochastic matrix, thus, $A^n$ is a row-stochastic matrix.  Therefore, all eigenvalues of $A$ satisfy $|\lambda| \leq 1$.



\end{proof}

\item[(b)] Show that
$$x^{(i)} \geq 0 \text{ and } \sum_{j=1}^n x_j^{(i)} = \sum_{j=1}^n x_j^{(0)} \text{ for all } i \geq 0.$$

\begin{itemize}
\item[(ii)] $x^{(i)} \geq 0$\\
\begin{proof} Since $A$ is row-stochastic, then $A^T$ is column-stochastic.\\

\underline{Base Case:}  Let $i = 0$.  Since we are given that $x_n^{(0)} \geq 0$ for all n, and $a^T_{jk} \geq 0$ for all $j,k = 1, \dots , n$ since it is column-stochastic.  Then, it follows that all entries of the product $A^T x^{(0)}$ are positive.  Thus, $x^{(1)} = A^T x^{(0)} \geq 0$.\\

\underline{Inductive Step:} By induction, we assume $x^{(i-1)} \geq 0$, and can show that $x^{(i)} \geq 0$.  Similarly to the base case, since $A^T$ is row-stochastic, then $a^T_{jk} \geq 0$ for all $j,k = 1, \dots , n$.  Then, it follows that all the entries of the product $A^T x^{(i-1)}$ are positive.  Thus, $x^{(i)} \geq 0$ for all $i=0,1,2, \dots$.\\

\end{proof}

\item[(ii)] $\sum_{j=1}^n x_j^{(i)} = \sum_{j=1}^n x_j^{(0)} \text{ for all } i \geq 0$\\
\begin{proof} $\text{ }$\\

\underline{Base Case:} Let $i = 0$.  Then,\\
\begin{align}
x^{(1)} &= A^T x^{(0)} \nonumber \\
&= \sum_{k=1}^n a^T_{jk} x^{(0)}_k \nonumber
\end{align}

Then, we have:\\
\begin{align}
\sum_{j=1}^n x^{(1)}_j &= \sum_{j=1}^n \left( \sum_{k=1}^n a^T_{jk} 			x^{(0)}_k\right) \nonumber \\
&= \sum_{k=1}^n \left( \sum_{j=1}^n a^T_{jk}\right)x^{(0)}_k \nonumber \\
&= \sum_{k=1}^n x^{(0)}_k. \nonumber
\end{align}
$\sum_{j=1}^n a^T_{jk} = 1$ since $A^T$ is column-stochastic.  Thus, it follows that
$$\sum_{j=1}^n x^{(1)}_j = \sum_{j=1}^n x^{(0)}_j.$$\\

\underline{Inductive Step:} By induction, assume:\\
$$\sum_{j=1}^n x^{(i-1)}_j = \sum_{j=1}^n x^{(i-2)}_j = \dots = \sum_{j=1}^n x^{(0)}_j.$$\\

Then,
\begin{align}
x^{(i)} &= A^T x^{(i-1)} \nonumber \\
&= \sum_{k=1}^n a^T_{jk} x^{(i-1)}_k \nonumber
\end{align}

Then, we have:
\begin{align}
\sum_{j=1}^n x^{(i)}_j &= \sum_{j=1}^n \left(\sum_{k=1}^n a^T_{jk} x^{(i-1)}_k \right) \nonumber \\
&= \sum_{k=1}^n \left(\sum_{j=1}^n a^T_{jk}\right) x^{(i-1)}_k \nonumber \\
&= \sum_{k=1}^n x^{(i-1)}_k \nonumber
\end{align}

Thus,\\
$$\sum_{j=1}^n x^{(i)}_j = \sum_{j=1}^n x^{(i-1)}_k.$$

Therefore, it follows that:
$$\sum_{j=1}^n x^{(i)}_j = \sum_{j=1}^n x^{(0)}_j.$$

\end{proof}

\end{itemize}

\item[(c)] Show that the vectors $x^{(i)}$ generated by the iteration
$$x^{(i+1)} = A^T x^{(i)}, \text{ } i=0,1,2, \dots $$
converge and that the limit
$$x = \lim_{i \rightarrow \infty} x^{(i)}$$
satisfies $A^T x = x$, $x \geq 0$, and $x \neq 0$.

\begin{proof} $\text{ }$ \\

\begin{align}
\lim_{i \rightarrow \infty} x^{(i)} &= \lim_{i \rightarrow \infty} 		\left(A^T x^{(i-1)}\right) \nonumber \\
&= \lim_{i \rightarrow \infty} \left(A^T\left(A^T x^{(i-2)}\right)		\right) \nonumber \\
& \vdots \nonumber \\
&= \lim_{i \rightarrow \infty} \left(A^T\right)^i x^{(0)} \nonumber 		\\
&= X\left(\lim_{i \rightarrow \infty} L^i \right) X^{-1} x^{(0)}, 		\nonumber
\end{align}
where

$$L = \begin{bmatrix}
		\lambda_1 & 0 & \hdots & 0 \\
		0 & \lambda_2 & \ddots & \vdots \\
		\vdots & \ddots & \ddots & 0 \\
		0 & \hdots & 0 & \lambda_n
		\end{bmatrix}. $$

Then,
\begin{align}
\lim_{i \rightarrow \infty} x^{(i)} &= X\left(\lim_{i \rightarrow 		\infty} L^i \right) X^{-1} x^{(0)} \nonumber \\
&= X \begin{bmatrix}
	1 & 0 & \hdots & 0 \\
	0 & 0 & \ddots & \vdots \\
	\vdots & \ddots & \ddots & 0 \\
	0 & \hdots & 0 & 0
	\end{bmatrix} X^{-1} x^{(0)}, \nonumber
\end{align}

where the product
$$X \begin{bmatrix}
	1 & 0 & \hdots & 0 \\
	0 & 0 & \ddots & \vdots \\
	\vdots & \ddots & \ddots & 0 \\
	0 & \hdots & 0 & 0
	\end{bmatrix} X^{-1} x^{(0)}$$
	
equals the first column of $X$, i.e., the eigenvector, $u_1$, corresponding to the dominant eigenvalue $|\lambda_1| = 1$.  So,

\begin{align}
\lim_{i \rightarrow \infty} x^{(i)} &= X \begin{bmatrix}
	1 & 0 & \hdots & 0 \\
	0 & 0 & \ddots & \vdots \\
	\vdots & \ddots & \ddots & 0 \\
	0 & \hdots & 0 & 0
	\end{bmatrix} X^{-1} x^{(0)} \nonumber \\
&= u_1 x^{(0)} \nonumber \\
&=: x \nonumber
\end{align}

Thus, the vectors generated by the iteration converge to $x$.\\

Now we need to show that $x$ satisfies the conditions $A^T x = x$, $x \geq 0$, and $x \neq 0$.  First, since $x = \lim_{i \rightarrow \infty} x^{(i)}$, and we have that $x^{(i+1)} = A^Tx^{(i)}$, then for $i$ large, $x^{(i+1)} \approx x^{(i)} = x$.  Thus, $A^T x = x$.  Second, we are given that $x^{(0)} \geq 0$ and $x^{(0)} \neq 0$, and we know that $A$ is row-stochastic, so it is non-negative.  Then, it follows that $x^{(1)} = A^Tx^{(0)} \geq 0$.  In part (b), we showed that $x^{(i)} \geq 0$ for all $i \geq 0$.  So clearly, $x = \lim_{i \rightarrow \infty} x^{(i)} \geq 0$.
\end{proof}
\end{itemize}

\bigskip\bigskip
\noindent
\textbf{Problem 4:} Let $G$ be a directed graph that describes the connectivity of a set n websites, and let $Q \in \R^{n\times N}$ and $A \in \R^{n\times n}$ be the associated matrices, as defined in class.

\begin{itemize}
\item[(a)] Write two Matlab functions, which use $n$, $Q$, and $J_v$ as inputs, to compute matrix-vector products of the form
$$y = A^Tx, \text{ where } x \in \R^n \text{ is a dense vector in general,}$$
and
$$y = A^T_{\alpha}x, \text{ where } x \in \R^n \text{ is a dense vector in general,}$$
respectively, as efficiently as possible.\\

\lstset{language=matlab,frame=single}
\begin{lstlisting}[caption=Matrix-Vector Product of Row Stochastic Matrix]
function [y] = RowStochastic(n,Q,Jv,x0)

e = ones(n,1);
v = zeros(n,1);

for i=1:length(Jv)
    v(Jv(i)) = 1;
end

v = sparse(v);
vT = transpose(v);
vTx0 = vT*x0;
QT = transpose(Q);

ATx0 = QT*x0 + (1/n)*e*vTx0;

y = ATx0;
end
\end{lstlisting}

\lstset{language=matlab,frame=single}
\begin{lstlisting}[caption=Matrix-Vector Product of Row Stochastic Matrix with $\alpha$]
function [yalpha] = RowStochasticAlpha(n,Q,Jv,alpha,x0)

e = ones(n,1);
v = zeros(n,1);

for i=1:length(Jv)
    v(Jv(i)) = 1;
end

v = sparse(v);
vT = transpose(v);
vTx0 = vT*x0;
QT = transpose(Q);
eTx0 = transpose(e)*x0;

ATx0 = QT*x0 + (1/n)*e*vTx0;
ATx0_alpha = alpha*ATx0 + (1 - alpha)*(1/n)*e*eTx0;

yalpha = ATx0_alpha;
end
\end{lstlisting}

\newpage
\underline{Results:}
	
\begin{table}[H]
\centering
\scalebox{.9}{%
\renewcommand{\arraystretch}{1.3}
\begin{tabular}{| c | c | c | c |}
\hline
 $A \in \R^{10\times 10}$ & $\alpha=1$ & $\alpha=0.5$ & $\alpha=0.85$ \\
\hline 
x & e & e & e \\
\hline
 y & $\begin{bmatrix}
	 5.000000000000000e-01 \\
     5.000000000000000e-01 \\
     7.500000000000000e-01 \\
     9.500000000000000e-01 \\
     8.000000000000000e-01 \\
     6.333333333333333e-01 \\
     3.116666666666666e+00 \\
     5.000000000000000e-01 \\
     1.416666666666667e+00 \\
     8.333333333333333e-01
     \end{bmatrix}$ & $\begin{bmatrix}
	 7.500000000000001e-01 \\
     7.500000000000000e-01 \\
     8.750000000000000e-01 \\
     9.750000000000001e-01 \\
     9.000000000000000e-01 \\
     8.166666666666667e-01 \\
     2.058333333333334e+00 \\
     7.500000000000000e-01 \\
     1.208333333333333e+00 \\
     9.166666666666667e-01
     \end{bmatrix}$ & $\begin{bmatrix}
		5.750000000000001e-01 \\
     	5.750000000000001e-01 \\
     	7.875000000000001e-01 \\
     	9.575000000000001e-01 \\
     	8.300000000000001e-01 \\
     	6.883333333333334e-01 \\
     	2.799166666666667e+00 \\
     	5.750000000000001e-01 \\
     	1.354166666666667e+00 \\
     	8.583333333333334e-01
     	\end{bmatrix}$  \\
\hline
\end{tabular}}
\end{table} 

\begin{table}[H]
\centering
\scalebox{1}{%
\renewcommand{\arraystretch}{1.3}
\begin{tabular}{| c | c | c |}
\hline
 $A \in \R^{685230\times 685230}$ & $\alpha=1$ &  $\alpha=0.85$ \\
\hline 
x & $x_0$ & $x_0$ \\
\hline
y(2) & -4.833020050815520e-02 & -4.108411686447250e-02 \\
y(222222)  & 1.715159442482511e-01 & 1.457851061784728e-01 \\
y(300000) & 9.711221038679012e-03 & 8.251091450336577e-03 \\
y(400000) & -7.835267507349471e-02 & -6.660322024501107e-02 \\
\hline
\end{tabular}}
\end{table} 
 

\item[(b)] Write a Matlab program that implements the power method, as stated in the notes provided on the course website.\\

\lstset{language=matlab,frame=single}
\begin{lstlisting}[caption=Power Method Function]
function [y] = PowerMethod(x, epsilon, kMax, n, Q, Jv, alpha)

lambda = [];
x_abs = abs(x);
[~,b] = max(x_abs);
X = x(b);
lambda(1) = X;

itCount = 0;

%%%
e = ones(n,1);
v = zeros(n,1);

for i=1:length(Jv)
    v(Jv(i)) = 1;
end

v = sparse(v);
vT = transpose(v);
QT = transpose(Q);
%%%

for k=2:kMax
    vTx = vT*x;
    ATx = QT*x + (1/n)*e*vTx;
    eTx = transpose(e)*x;
    x = alpha*ATx + (1-alpha)*(1/n)*e*eTx;
    x_abs = abs(x);
    %[~,b] = sort(x_abs,'descend');
    [~,b] = max(x_abs);
    X = x(b);
    lambda(k) = X;
    x = x/lambda(k);
    itCount = itCount + 1;
    
    if (abs(lambda(k) - lambda(k-1)) <= epsilon*(abs(lambda(k-1))))
        break
    end
end

itCount = itCount
evalApprox = lambda(itCount+1)
evecApprox = x;
vTx = vT*x;
ATx = QT*x + (1/n)*e*vTx;
eTx = transpose(e)*x;
ATalphax = alpha*ATx + (1-alpha)*(1/n)*e*eTx;
relEigResidual = norm((ATalphax - evalApprox*x),inf)/norm(x,inf)
v = x;
[~,d] = sort(v, 'descend');
pageRank = d(1:10)
end
\end{lstlisting}
*note: I know that it would be simpler to call my function RowStochasticAlpha within my PowerMethod function.  However, I was having difficulties getting my code to work doing it that way so I essentially just put everything from my RowStochasticAlpha function into my PowerMethod function, and it works that same as if I had called the RowStochasticAlpha function. \\

\item[(c)] Use your program from (b) to compute the dominate eigenvalue $\lambda$ and a corresponding eigenvector $x$ of the matrix $A^T$ corresponding to the case of $n=10$ websites stated in Problem 2.\\

\begin{table}[H]
\centering
\scalebox{1}{%
\renewcommand{\arraystretch}{1.3}
\begin{tabular}{| c | c | c |}
\hline
 &  $x=e$ & $x=[1; -1; \dots; 1; -1]$ \\
\hline 
$\alpha$ & 1 & 1 \\
\hline
Iterations & 47 & 100 \\
$\lambda$ & 1.000000000000000e+00 & 9.999999999999999e-01 \\
Residual & 5.551115123125783e-16 & 1.665334536937735e-16 \\
$x$ & $\begin{bmatrix}
			2.154770537656686e-01 \\
     		2.386822749404329e-01 \\
     		2.751476225007769e-01 \\
     		3.633067440174038e-01 \\
     		2.582616803066403e-01 \\
     		2.300839117372837e-01 \\
     		1.000000000000000e+00 \\
     		2.386822749404329e-01 \\
     		4.407956075831347e-01 \\
     		3.247695017093130e-01  
     		\end{bmatrix} $ & $\begin{bmatrix}
		    2.154770537656687e-01 \\
     		2.386822749404331e-01 \\
     		2.751476225007770e-01 \\
    			3.633067440174039e-01 \\
     		2.582616803066405e-01 \\
     		2.300839117372837e-01 \\
     		1.000000000000000e+00 \\
     		2.386822749404331e-01 \\
     		4.407956075831347e-01 \\
     		3.247695017093132e-01  
    		\end{bmatrix}$\\
Page Rank & 9, 4, 10, 3, 5, 2, 8, 6, 1, 7 & 9, 4, 10, 3, 5, 2, 8, 6, 1, 7 \\
\hline
\end{tabular}}
\end{table} 
*Note: Websites 2 and 8 have the same page rank, so their order is interchangeable.\\


\item[(d)] Use your program from (b) to try to compute the dominate eigenvalue $\lambda$ and a corresponding eigenvector $x$ of the matrix $A^T$ corresponding to the large graph with $n=685230$ nodes and $7600595$ edges given in "www0.mat".\\


\begin{table}[H]
\centering
\scalebox{0.8}{%
\renewcommand{\arraystretch}{1.3}
\begin{tabular}{| c | c | c | c | c |}
\hline
 &  $x=e$ & $x=x_0$ & $x=e$ & $x=x_0$ \\
\hline 
$\alpha$ & 1 & 1 & 0.85 & 0.85 \\
\hline
Iterations & 9999 & 9999 & 122 & 163 \\
$\lambda$ & 1.000000895792460e+00 & 1.124042805557909e+00 & 				9.999999999952399e-01 & 1.000000000004515e+00 \\
Residual & 2.029381492183098e-01 & 3.622262713234378e-01 & 				1.105315166650966e-10 & 1.688181712478087e-10 \\
Page Rank & $\begin{bmatrix}
			403967 \\
      		333816 \\
      		407349 \\
      		158244 \\
      		208355 \\
      		310294 \\
      		115404 \\
      		463748 \\
      		569443 \\
      		525917
      		\end{bmatrix}$
& $\begin{bmatrix}
				407349 \\
				158244 \\
				553109 \\
				505281 \\
				487780 \\
				569443 \\
				525917 \\
				665540 \\
				187367 \\
				382960
				\end{bmatrix}$ & $\begin{bmatrix}
     		 629103 \\
      		328995 \\
      		176090 \\
      		  5397 \\
      		609117 \\
      		539881 \\
      		 63085 \\
      		351516 \\
      		529968 \\
      		363332
      		\end{bmatrix}$ & $\begin{bmatrix}
     		 629103 \\
      		328995 \\
      		176090 \\
      		  5397 \\
      		609117 \\
      		539881 \\
      		 63085 \\
      		351516 \\
      		529968 \\
      		363332
      		\end{bmatrix}$ \\
\hline
\end{tabular}}
\end{table} 

For the last two cases, when $\alpha = 0.85$, the convergence is very fast.  Unlike the first two cases, when $\alpha = 1$, that don't converge within the maximum number of iterations.  So, the page rankings for the $\alpha=1$ cases are meaningless.  Whereas when $\alpha = 0.85$, the page ranking is the same regardless of what our initial value of $x$ was.\\

\end{itemize}


\bigskip\bigskip
\noindent
\textbf{Problem 5:} 
\begin{itemize}
\item[(a)] Show that all $n$ eigenvalues of an $n\times n$ circulant matrix $C$ can be computed via a single call to Matlab's "fft" command.\\
\begin{proof} $\text{ }$\\
Let $H$ be the Hermitian operator, such that 
$$A^H := \overline{A}^T = \overline{A^T}.$$\\
Note that since $\overline{F}$ is symmetric, then
$$\overline{F}^T = \overline{F}.$$\\
In the problem statement, the eigenvalues are defined as:
$$\lambda_k = \sum_{j=0}^{n-1} c_j exp \left(\frac{2\pi i}{n}jk \right),$$
which can also be written as:
$$\Lambda = \overline{F}c = \overline{F}^Tc,$$
since $\overline{F}$ is symmetric.  Then,

\begin{align}
\overline{F}c &= \overline{F}^Tc \nonumber \\
&= F^Hc \nonumber \\
&= \left(c^H F\right)^H \nonumber \\
&= \left( c^T F \right)^H \nonumber \\
&= \left(\left(F^T c \right)^T\right)^H \nonumber \\
&= \overline{\left(\left(Fc\right)^T\right)^T} \nonumber \\
&= \overline{(Fc)} 
\end{align}

So, combining (2) with our result from part (a), we have that 
$$\Lambda = \overline{F}c = \overline{\text{fft}(c)}.$$
Thus, the eigenvalues of the circulant matrix $C$ can be computed in Matlab with the command
$$\text{conj(fft(c))}.$$

\end{proof}

\item[(b)] Use part (a) and the factorization $C = \frac{1}{n}\overline{F}\Lambda F$ to show that each matrix-vector product $y = Cx$ with a circulant matrix $C$ can be computed via three calls to Matlab's "fft" command.\\
Write a Matlab function that implements this approach to efficiently compute matrix-vector products with circulant matrices.  Use your program to compute $y=Cx$ for
$$C = \begin{bmatrix}
		2 & -1 & 0 & -3 \\
		-3 & 2 & -1 & 0 \\
		0 & -3 & 2 & -1 \\
		-1 & 0 & -3 & 2 
		\end{bmatrix}
\text{ and } x = \begin{bmatrix}
				-1 \\
				2 \\
				1 \\
				4
				\end{bmatrix}.$$\\
\begin{proof} $\text{ }$\\
\begin{align}
Cx &= \frac{1}{n}\overline{F}\Lambda Fx \nonumber \\
&= \frac{1}{n}\overline{F}(\text{conj(fft(c))})Fx \nonumber \\
&= \frac{1}{n}\overline{F}(\text{conj(fft(c))})(\text{fft(x)}) 			\nonumber \\
&= \frac{1}{n} \text{conj[fft(conj(fft(c))(fft(x)))]} \nonumber
\end{align}
\end{proof}

\lstset{language=matlab,frame=single}
\begin{lstlisting}[caption=Circulant Matrix-Vector Product Function]
function [y_fast] = fftCirculant(c,x)

n = length(x);
 
lambdaVec = conj(fft(c));
 
y_exact = C*x;
y_fast = (1/n)*conj(fft(conj(lambdaVec.*fft(x))));
end
\end{lstlisting}


 \underline{Results:}\\
 $$y_{\text{exact}} = \begin{bmatrix}
 			-16 \\
 			6 \\
 			-8 \\
 			6 
 			\end{bmatrix}$$
 
 $$y_{\text{fast}} = \begin{bmatrix}
 			-16 \\
 			6 \\
 			-8 \\
 			6 
 			\end{bmatrix}$$\\
 
 
 \item[(c)] Show that for any given Toeplitz matrix $T \in \R^{n\times n}$ there is a unique circulant matrix $C \in \R^{(2n-1)\times(2n-1)}$ such that $T$ is the leading principal $n\times n$ submatrix of $C$.\\
 
 \begin{proof} $\text{ }$\\
 The Toeplitz matrix $T$ is defined as 
 $$T = [t_{k-j}]_{j,k=1,2,\dots ,n} \in \R^{n\times n},$$ 
 where 
 $$t:= \begin{bmatrix}
 		t_{-(n-1)} & t_{-(n-2)} & \hdots & t_{-1} & t_0 & t_1 & \hdots & t_{n-1}
 		\end{bmatrix}.$$\\
 		
So, our Toeplitz matrix $T$ is of the form:\\
$$T = \begin{bmatrix}
	t_0 & t_1 & t_2 & \hdots & t_{n-1} \\
	t_1 & t_0 & t_1 & \hdots & t_{n-2} \\
	\vdots & \ddots & \ddots &  & \vdots \\
	t_{-(n-2)} & t_{-(n-3)} & \hdots & \hdots & \vdots \\
	t_{-(n-1)} & t_{-(n-2)} & \hdots & \hdots & t_0
	\end{bmatrix}.$$\\

Since we want to show that for any given Toeplitz matrix $T \in \R^{n\times n}$ there is a unique circulant matrix $C \in \R^{(2n-1)\times (2n-1)}$ such that $T$ is the leading principal $n\times n$ submatrix of $C$, then we need to redefine $t$ in terms of $c$, where
$$c:= \begin{bmatrix}
		c_0 & c_1 & c_2 & \hdots & c_{n-1}
		\end{bmatrix},$$
as follows:\\
$$ t = \begin{bmatrix}
		c_n & c_{n+1} & \hdots & c_{2n-1} & c_0 & c_1 & c_2 & \hdots & c_{n-1}
		\end{bmatrix}.$$\\

Now, we can write our circulant matrix $C$, with leading submatrix $T$ as:\\

$$ C = \begin{bmatrix}
		c_0 & c_1 & c_2 & \hdots & c_{n-1} &\vline & c_n & \hdots & c_{2n-2} & c_{2n-1} \\
		c_{2n-1} & c_0 & c_1 & \hdots & c_{n-2} & \vline & c_{n-1} & \hdots & c_{2n-3} & c_{2n-2} \\
		c_{2n-2} & c_{2n-1} & c_0 & \hdots & c_{n-3} & \vline & c_{n-2} & \hdots & c_{2n-4} & c_{2n-3} \\
		\vdots & \ddots & \ddots & & \vdots & \vline & \vdots & \ddots & & \vdots \\
		c_n & c_{n+1} & c_{n+2} & \hdots & c_0 & \vline &  \vdots & \ddots & \ddots & \vdots \\
		\hline 
		c_{n-1} & c_n & c_{n+1} & \hdots & c_1 & \vline & \vdots & \ddots & \ddots & \vdots \\
		\vdots & \ddots & \ddots & & \vdots & \vline & \vdots & \ddots & \ddots & \vdots \\
		c_2 & \hdots & \ddots & \ddots & \vdots & \vline & \vdots & \ddots & \ddots & \vdots \\
		c_1 & c_2 & c_3 & \hdots & c_n & \vline & c_{n+1} & \hdots & \hdots & c_0
		\end{bmatrix}.$$\\
		
Where the upper left block is the $n\times n$ Toeplitz matrix $T$, expressed in terms of $c$.  The lower left block is $(n-1)\times n$, the upper right block is $n\times(n-1)$, and the lower right block is $(n-1) \times (n-1)$.  So, $C$ is $(2n-1) \times (2n-1)$.\\

Now, we need to check that the property of Toeplitz matrices,
$$t_{-j} = t_{n-j} \text{ for all } j=1,2, \dots , n-1,$$
holds when $t$ is written in terms of $c$, as defined above.\\

First, let's look at the case of $j=1$.  Since $T$ is Toeplits, then $$t_{-1} = t_{n-1}.$$  
So, in terms of $c$, this equates to 
$$c_{2n-1} = c_{n-1}.$$  
Which we can see are in the correct spots of the Toeplitz block (upper left block) of our circulant matrix $C$, so this property holds for $j=1$.\\

Next, looking a the case of $j=n-1$, then we have that 
$$t_{-(n-1)} = t_1$$
corresponds to 
$$c_n = c_1.$$
Which we can see, as before, are in the correct positions of our matrix $C$.\\
So, in general, if 
$$t_{-j} = t_{n-j},$$
then
$$c_{2n-j} = c_{n-j} \text{ for all } j=1, 2, \dots , n-1.$$\\

Thus, for any given Toeplitz matrix $T$ there is a unique circulant matrix $C$ such that $T$ is the leading principal $n\times n$ submatrix of $C$.
		
 \end{proof}
 \bigskip
 \item[(d)] Use (c) to devise an algorithm that computes matrix-vector products $y=Tx$ with Toeplitz matrices $T$ via your Matlab function from (b).\\
 Write a Matlab function that implements this approach.\\
 
\lstset{language=matlab,frame=single}
\begin{lstlisting}[caption=Toeplitz Matrix-Vector Product Function]
function [y] = fftToeplitz(t,x)

n = length(x);

c = zeros(1,2*n-1);

c(1:n) = t(n:2*n-1);
c(n+1:2*n-1) = t(1:n-1);

x_tilde = zeros(2*n-1,1);

for i = 1:n
    x_tilde(i) = x(i);
end

c = c';

y_tilde = fftCirculant(c,x_tilde);

y = y_tilde(1:n);

end
\end{lstlisting}
 $\text{ }$ \\
 
 \item[(e)] $\text{ }$\\
 \begin{itemize}
 \item[(i)] To test your program from (d), first employ it to compute $y=Tx$ for 
 $$T = \begin{bmatrix}
 		2 & -1 & 4 & -3 \\
 		1 & 2 & -1 & 4 \\
 		-5 & 1 & 2 & -1 \\
 		6 & -5 & 1 & 2
 		\end{bmatrix}
 \text{ and } x = \begin{bmatrix}
 					-1 \\
 					2 \\
 					1 \\
 					4
 					\end{bmatrix}.$$\\
 
\underline{Results:} \\
$$y_{\text{exact}} = \begin{bmatrix}
						-12 \\
						18 \\
						5 \\
						-7
						\end{bmatrix}, \text{ } y_{\text{fast}} = \begin{bmatrix}
					-1.199999999999999e+01 \\
     				1.800000000000000e+01 \\
     				5.000000000000000e+00 \\
    					-7.000000000000001e+00 
    					\end{bmatrix}$$\\

\item[(ii)] Now employ your program to compute $y=Tx$ where $T$ is the $n\times n$ Toeplitz matrix given by the row vector $t$.\\
 
 
\begin{table}[H]
\centering
\renewcommand{\arraystretch}{1.3}
\begin{tabular}{| c | c |}
\hline
Result &  Value\\
\hline 
$y_1$ & 4.743400116337570e+01 - 1.897571643350681e-14i\\

$y_{100000}$ &  -2.044536337994736e+02 + 3.463356617877540e-14i \\

$y_{200000}$ & 9.854729389973313e+01 - 2.893191580611515e-14i \\

$y_{300000}$ & -2.273143080393689e+01 + 4.576858382960005e-14i\\

$y_{400000}$ &  4.136625635601638e+02 - 5.480763011145770e-14i \\

$y_{500000}$ & 2.524304751530099e+02 + 9.171369375379089e-14i \\

$\displaystyle \sum_{j=1}^n y_j $ &  1.030143367304306e+05 + 7.836054370961585e-13i\\

$\|y\|_2$ &  1.669299439051479e+05\\
\hline
\end{tabular}
\end{table} 
 

 
\end{itemize}
 
 
\end{itemize}

\end{document}

%\definecolor{mypink1}{rgb}{0.858, 0.188, 0.478}
%\definecolor{Mycolor2}{HTML}{00F9DE}
%$\color{Mycolor2}\heartsuit$
%\rightthumbsup
